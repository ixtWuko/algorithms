\section{基础知识}
\subsubsection*{渐进记号}
\begin{itemize}
    \item $\Theta$: $f(n) = cg(n)$
    \item $O$: $f(n) \le cg(n)$
    \item $o$: $f(n) < cg(n)$
    \item $\Omega$: $f(n) \ge cg(n)$
    \item $\omega$: $f(n) > cg(n)$
\end{itemize}
这份笔记中的运行时间基本都是指最坏情况运行时间。

\subsubsection*{求解递归式}
递归式用于分析分治算法的运行时间。
\begin{itemize}
    \item 代入法:猜测一个界,然后验证。
    \item 递归树法:将递归式转换为一棵树,求和。
    \item 主方法:使用主定理

    若递归式$T(n) = aT(n/b) + f(n)$中,$a \ge 0, b > 0$,则比较$f(n)$与$n^{\log _b a}$,
    \begin{itemize}
        \item $f(n) > n^{\log _b a}$, $T(n) = \Theta (n^{\log _b a})$.
        \item $f(n) < n^{\log _b a}$, $T(n) = \Theta (f(n))$.
        \item $f(n) = n^{\log _b a}$, $T(n) = \Theta (n^{\log _b a} \log _2 n)$.
    \end{itemize}

    另外还有 Akra-Bazzi 方法。
\end{itemize}

\subsubsection*{概率分析}
概率分析用于分析随机算法的运行时间。

指示器随机变量:$I\{A\} = \begin{cases} 1 & A \text{发生}\\ 0 & A \text{不发生} \end{cases}$

\begin{itemize}
    \item 对于一次事件:$E(I\{A\}) = 1 \cdot P(A) + 0 \cdot P(\overline A) = P(A)$
    \item 对于$n$次事件:$E(X) = E(\displaystyle \sum _{i=1}^n X_i) = \sum _{i=1}^n E(I\{A_i\}) = \sum _{i=1}^n P(A_i)$
\end{itemize}

产生一个均匀随机序列:指能够等可能地产生包含所有元素的每一种序列的方式,随机产生的一个序列。

\section{数据结构}

\section{搜索}
\begin{itemize}
    \item 线性搜索 linear search: $\Theta (n)$
    \item 二分搜索 binary search: $\Theta (\log n)$
    \item 随机搜索 random search: $\Theta (n)$
\end{itemize}

\section{排序}
\begin{itemize}
    \item 选择排序 selection sort: $\Theta (n^2)$
    \item 插入排序 insertion sort: $\Theta (n^2)$
    \item 冒泡排序 bubble sort: $\Theta (n^2)$
    \item 归并排序 merge sort: $\Theta (n \log n)$
\end{itemize}